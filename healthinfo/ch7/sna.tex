\documentclass[11pt]{beamer}
\usetheme[]{CambridgeUS}

%-------------------------------------------------------
% INCLUDE PACKAGES
%-------------------------------------------------------

\usepackage[utf8]{inputenc}
\usepackage{CJKutf8}
\usepackage[english]{babel}
\usepackage[T1]{fontenc}
\usepackage{helvet}
\usepackage{caption}

%-------------------------------------------------------
% INFORMATION IN THE TITLE PAGE
%-------------------------------------------------------

\title[] % [] is optional - is placed on the bottom of the sidebar on every slide
{ % is placed on the title page
      \LARGE \textbf{第七章 社会网络分析法}
}

\subtitle[]
{
      \textbf{Social Network Analysis}
}

\author[]
{     \hspace{-7em} 授课教师: 吴翔 \\
	  \vspace{1em}
      邮箱: wuhsiang@hust.edu.cn
}

\date{Oct 24, 2018}

%-------------------------------------------------------
% THE BODY OF THE PRESENTATION
%-------------------------------------------------------

\begin{document}
\begin{CJK}{UTF8}{gbsn}

%-------------------------------------------------------
% THE TITLEPAGE
%-------------------------------------------------------

\begin{frame}[plain,noframenumbering] % the plain option removes the header from the title page, noframenumbering removes the numbering of this frame only
  \titlepage % call the title page information from above
\end{frame}


\begin{frame}{目录}{}
\tableofcontents
\end{frame}

\begin{frame}

\begin{itemize}
	\item 课件存储地址: \href{https://github.com/wuhsiang/Courses}{https://github.com/wuhsiang/Courses}
	\centering
	\includegraphics[width=0.3\textwidth]{../../QR.png}
\end{itemize}

\end{frame}


%-------------------------------------------------------
\section{1 SNA概述}
%-------------------------------------------------------

\begin{frame}[plain]

\begin{center}
	\Huge 1 SNA概述
\end{center}

\end{frame}

% \begin{frame}{1.2 古典概率}
%
% \begin{block}{古典概率(classical probability)}
% 在相同条件下进行了$n$次\textcolor{red}{随机试验},其中事件A发生的次数(即频数)为$n_{A}$。换言之,事件A发生的\textcolor{red}{频率}为$f_{n}(A) = n_{A} / n$。古典概率$p(A)$定义为
% $$
% p(A)= lim_{n\to\infty} f_{n}(A)
% $$
% 例子:抛硬币
% \end{block}
%
% 考虑以下情境:
%
% \begin{itemize}
% 	\item 一百年内发生第三次世界大战的概率?
%     \item 医疗检测中,受检者患病的概率?
% \end{itemize}
%
% \end{frame}
%
%
% \begin{frame}{1.3 先验和后验概率}
%
% \begin{block}{先验和后验概率(prior and posterior probability)}
% \begin{enumerate}
% \item 先验概率:获得观测数据$D$前,表达随机事件$A$不确定性的概率$p(A)$
% \item 后验概率:获得观测数据$D$后,表达随机事件$A$不确定性的概率$p(A|D)$
% \end{enumerate}
% \end{block}
%
% \begin{figure}[H]
% \begin{minipage}[b]{.48\linewidth}
% \centering
% \begin{tabular}{|c|c|} \hline
%      随机事件 & 先验概率 \\ \hline
%       晴天    & 0.3 \\
%       阴天    & 0.5 \\
%       雨天    & 0.2 \\ \hline
%     \end{tabular}
%     \captionof{table}{离散随机变量}
% \end{minipage}
% \hfill
% \begin{minipage}[b]{0.48\linewidth}
%  \centerline{\includegraphics[width=0.8\textwidth]{Feathergraphics/dprior.jpeg}}
%  \caption{连续随机变量}
% \end{minipage}
% \end{figure}
%
% \end{frame}


%-------------------------------------------------------
\section{2 SNA的主要分析角度}
%-------------------------------------------------------

\begin{frame}[plain]

\begin{center}
	\Huge 2 SNA的主要分析角度
\end{center}

\end{frame}


%-------------------------------------------------------
\section{3 SNA的主要软件工具}
%-------------------------------------------------------

\begin{frame}[plain]

\begin{center}
	\Huge 3 SNA的主要软件工具
\end{center}

\end{frame}


%-------------------------------------------------------
\section{4 SNA在医学领域的应用}

\begin{frame}[plain]

\begin{center}
	\Huge 4 SNA在医学领域的应用
\end{center}

\end{frame}



\end{CJK}

\end{document}
